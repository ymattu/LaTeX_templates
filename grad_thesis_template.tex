\documentclass[11pt]{jsarticle}

\usepackage{url}
\usepackage{natbib}
\usepackage{color}
\usepackage[dvipdfmx]{graphicx}
\usepackage{here}
\usepackage{amsmath}
\usepackage{amsfonts}
\usepackage{amssymb}
\usepackage{bm}
\usepackage{multirow}
%\usepackage{tabudlarx}
\usepackage{latexsym}

\usepackage{mathrsfs}%花文字のため

%argmax,argmin の定義
\DeclareMathOperator*{\argmin}{arg\,min}
\DeclareMathOperator*{\argmax}{arg\,max}



\title{タイトル \\ 長いときは改行}
\author{学籍番号 1111111 \\ ymattu}
\date{2017 年 2 月 3 日}

\begin{document}


%%% 表紙(タイトルページ)はじまり %%%
\begin{titlepage}
\begin{center}

 \LARGE{ 卒業論文} \hspace{\fill} \LARGE{2020 年度}\\

  \vspace{3cm}
 {\fontsize{20pt}{20pt}\selectfont
    タイトル \\ 長いときは改行
  % vspace*{0.5cm}
  }\\ % 副題を書く

  \vspace{2.5cm}
 {\fontsize{20pt}{20pt}\selectfont  ymattu} \\
 \LARGE{(学籍番号:1111111)} \\
  \vspace{3.5cm}
 \LARGE{指導教員 教授 大 魔神} \\
  \vspace{2cm}
 \LARGE{慶應義塾大学経済学部} \\
 \LARGE{経済学科} \\

\end{center}
\end{titlepage}


\newpage
 \thispagestyle{empty}

\begin{abstract}
 要旨をコピペでいいんじゃない?
\end{abstract}

\begin{center}
 キーワード: なんか、あれば
\end{center}

\newpage
% 目次

 \pagenumbering{roman}
 % setcounter{tocdepth}{3} %subsubsection まで目次に表示
\tableofcontents

\newpage
% 図表目次
\listoffigures
\listoftables

\newpage
 \pagenumbering{arabic}
 \setcounter{page}{1}

\section{はじめに}

\section{先行研究}
\citet{rosenbaum1983central}によると

\section{モデル}

\section{実証分析}

\section{まとめ}


\newpage
\bibliographystyle{jecon}
\bibliography{sample}

\end{document}


